\documentclass{supervision}
\usepackage{course}

\Supervision{3}
\begin{document}
  \begin{questions}
    \SetQuestionNumber{20}
    \question Read Ken Thompson: \textit{Reflections on Trustung Trust},
      Communications of the ACM, Vol 27, No 8, August 1984, pp 761-763 and
      explain how even a careful inspection of all source code within the TCB
      might miss carefully planted backdoors.

    \question You are a technician working for the intelligence agency of
      Amoria. Your employer is extremely curious about what goes on in a
      particular ministry of Bumaria. This ministry has ordered networked
      computers from an Amorian supplier and you will be given access to the
      shipment before it reaches the customer. What modifications could you
      perform on the hardware to help with later break-in attempts, knowing that
      the Bumarian government only uses software from sources over which you
      have no control?

    \question The Bumarian government is forced to buy Amorian computers as its
      national hardware industry is far from competitive. However, there are
      strong suspicions that the Amorian intelligence agencies regularly modify
      hardware shipments to help in their espionage efforts. Bumaria has no lack
      of software skills and the government uses its own operating system.
      Suggest to the Bumarians some operating system techniques that can reduce
      the information security risks of potential malicious hardware
      modifications.

    \SetQuestionNumber{32}
    \question The log file of your HTTP server shows odd requests such as
      \begin{code}{{}}
        GET /scripts/..%255c..%255cwinnt/system32/cmd.exe?/c+dir+C:\
        GET /scripts/..%u002f..%u002fwinnt/system32/cmd.exe?/c+dir+C:\
        GET /scripts/..%e0%80%af../winnt/system32/cmd.exe?/c+dir+C:\
      \end{code}

      Explain the attacker’s exact flaw hypothesis and what these penetration
      attempts try to exploit.

      (Is there a connection with the floor tile pattern outside the lecture
      theatre?)

    \section*{2004 Paper 3 Question 9}
    \question Introduction to Security
      \begin{parts}
        \part[5] Explain briefly mechanisms that software on a desktop computer
          can use to securely generate secret keys for use in cryptographic
          protocols.


        \part[5] Give \emph{two} different ways of implementing residual
          information protection in an operating system and explain the threat
          addressed by each.

        \part Consider the standard POSIX file-system access control mechanism:
          \begin{subparts}
            \subpart[2] Under which conditions can files and subdirectories be
              removed from a parent directory?

            \subpart[2] Many Unix variants implement an extension known as the
              ``sticky bit''. What is its function?

            \subpart[2] On a POSIX system that lacks support for the ``sticky
              bit'', how could you achieve an equivalent effect?
          \end{subparts}

        \part[4] VerySafe Ltd offer two vaults with electronic locks. They open
          only after the correct decimal code has been entered. The VS100 – a
          low-cost civilian model – expects a 6-digit code. After all six digits
          have been entered, it will either open or will signal that the code
          was wrong and ask for another try. The VS110 – a far more expensive
          government version – expects a 40-digit code. Users of a beta-test
          version of the VS110 complained about the difficulty of entering such
          a long code correctly. The manufacturer therefore made a last-minute
          modification. After every five digits, the VS110 now either confirms
          that the code has been entered correctly so far, or it asks for the
          previous five digits again. Compare the security of the VS100 and
          VS110.
      \end{parts}

    \question Do software updates increase or decrease risk of successful
      attack?

    \section*{Introduction to Modern Cryptography Chapter 3}
    \SetQuestionNumber{21}
    \question Let $\prod_1 = ({Gen}_1, {Enc}_1, {Dec}_1)$ and $\prod_2 =
      ({Gen}_2, {Enc}_2, {Dec}_2)$ be two encryption schemes for which it is
      known that at least one is CPA-secure. The problem is that you don't know
      which one is CPA-secure and which one may not be. Show how to construct an
      encryption scheme $\prod$ that is guaranteed to be CPA-secure as long as
      at least one of $\prod_1$ or· $\prod_2$ is CPA-secure. Try to provide a
      full proof of your answer.

      % Hint: Generate two plaintext messages from the original plaintext so
      % that knowledge of either one of the parts reveals nothing about the
      % plain­ text, but knowledge of both does yield the original plaintext.

  \end{questions}
\end{document}
