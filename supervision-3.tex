\documentclass{supervision}
\usepackage{course}

\Supervision{3}
\begin{document}
  \begin{questions}
    \SetQuestionNumber{20}
    \question Read Ken Thompson: \textit{Reflections on Trusting Trust},
      Communications of the ACM, Vol 27, No 8, August 1984, pp 761-763 and
      explain how even a careful inspection of all source code within the TCB
      might miss carefully planted backdoors.
      \begin{solution}
        Inspecting the source only helps you if you then compile from the
        source, otherwise there is no guarantee that the binary on your machine
        is made from the same source you are inspecting.

        Compilation requires a compiler, since a rogue compiler can simply
        add a trojan to any program it compiles it is necessary to trust the
        compiler. Inspecting the source of the compiler is clearly not
        sufficient as the source will be compiled with the compiler itself.
      \end{solution}

    \question You are a technician working for the intelligence agency of
      Amoria. Your employer is extremely curious about what goes on in a
      particular ministry of Bumaria. This ministry has ordered networked
      computers from an Amorian supplier and you will be given access to the
      shipment before it reaches the customer. What modifications could you
      perform on the hardware to help with later break-in attempts, knowing that
      the Bumarian government only uses software from sources over which you
      have no control?
      \begin{solution}
        Depending on how much comparative value is placed on remaining covert
        and gaining intelligence I would perform some of the following:

        \begin{itemize}
          \item Install a GSM keylogger in the keyboard
          \item Install custom firmware on the network chip such that whenever
            packets are sent to an IP address that belongs to Amoria the unused
            IP fields are used to carry some data from disk or ram for
            interception.
          \item Install custom firmware on the hard drive such that when a
            sufficiently random string is encountered it will, for the next $x$
            seconds give some predetermined value for the password file.
        \end{itemize}
      \end{solution}

    \question The Bumarian government is forced to buy Amorian computers as its
      national hardware industry is far from competitive. However, there are
      strong suspicions that the Amorian intelligence agencies regularly modify
      hardware shipments to help in their espionage efforts. Bumaria has no lack
      of software skills and the government uses its own operating system.
      Suggest to the Bumarians some operating system techniques that can reduce
      the information security risks of potential malicious hardware
      modifications.
      \begin{solution}
        I'm not sure what an operating system can possible do to protect against
        completely compromised hardware. It can't put anything secret in RAM
        since that can be compromised and yet at some point the data would have
        to be decrypted so that it could be used.

        I did think you could have a permutation function over the memory
        addresses to randomize it, but such a function would itself have to be
        stored in RAM.
      \end{solution}

    \SetQuestionNumber{32}
    \question The log file of your HTTP server shows odd requests such as
      \begin{code}{{}}
        GET /scripts/..%255c..%255cwinnt/system32/cmd.exe?/c+dir+C:\
        GET /scripts/..%u002f..%u002fwinnt/system32/cmd.exe?/c+dir+C:\
        GET /scripts/..%e0%80%af../winnt/system32/cmd.exe?/c+dir+C:\
      \end{code}

      Explain the attacker’s exact flaw hypothesis and what these penetration
      attempts try to exploit.

      (Is there a connection with the floor tile pattern outside the lecture
      theatre?)
      \begin{solution}
        The attacker is checking to see whether the web server allows traversing
        up a directory and whether it parses non-standard unicode escape
        sequences as forward slashes.

        If it does and the webserver simply executes the script referenced by
        the URL then this will execute a command prompt with a command to list
        the directory contents of the root directory of the C drive.
      \end{solution}

    \section*{2004 Paper 3 Question 9}
    \question Introduction to Security
      \begin{parts}
        \part[5] Explain briefly mechanisms that software on a desktop computer
          can use to securely generate secret keys for use in cryptographic
          protocols.
          \begin{solution}
            It can take multiple sources of random data like time between
            interrupts, key presses, mouse movement etc. to create an entropy
            pool from which a pseudo random bit stream generator is seeded.
          \end{solution}

        \part[5] Give \emph{two} different ways of implementing residual
          information protection in an operating system and explain the threat
          addressed by each.
          \begin{solution}
            \emph{Doesn't appear to be on the course anymore}

            \begin{description}
              \item[Secure delete files] When a file is deleted then overwrite
                it several times to ensure that the data could not be accessed
                by a third party later with access to the drive.

              \item[Zero registers on switch] When the operating system performs
                a context switch it should zero the registers to prevent data
                leaking from one process to another.
            \end{description}
          \end{solution}

        \part Consider the standard POSIX file-system access control mechanism:
          \begin{subparts}
            \subpart[2] Under which conditions can files and subdirectories be
              removed from a parent directory?
              \begin{solution}
                If write access for the parent directory is present.
                If Write access for the file being deleted is present.
              \end{solution}

            \subpart[2] Many Unix variants implement an extension known as the
              ``sticky bit''. What is its function?
              \begin{solution}
                When the sticky bit is set, only the file/directory owner or
                root can rename or delete the file.
              \end{solution}

            \subpart[2] On a POSIX system that lacks support for the ``sticky
              bit'', how could you achieve an equivalent effect?
              \begin{solution}
                Not sure.
              \end{solution}
          \end{subparts}

        \part[4] VerySafe Ltd offer two vaults with electronic locks. They open
          only after the correct decimal code has been entered. The VS100 – a
          low-cost civilian model – expects a 6-digit code. After all six digits
          have been entered, it will either open or will signal that the code
          was wrong and ask for another try. The VS110 – a far more expensive
          government version – expects a 40-digit code. Users of a beta-test
          version of the VS110 complained about the difficulty of entering such
          a long code correctly. The manufacturer therefore made a last-minute
          modification. After every five digits, the VS110 now either confirms
          that the code has been entered correctly so far, or it asks for the
          previous five digits again. Compare the security of the VS100 and
          VS110.
          \begin{solution}
            The VS100 requires a six digit code and thus there are $10^6$
            different permutations.

            The VS110 expects 8 5-digit passcodes and thus a user only has to
            try $10^5$ combinations for each block and thus a total of $8 \times
            10^5$. Therefore the VS100 provides more security.
          \end{solution}
      \end{parts}

    \question Do software updates increase or decrease risk of successful
      attack?
      \begin{solution}
        Software updates provide a useful mechanism to keep software up to date
        with latest patches which prevent security vulnerabilities from being
        exploited. The update procedure is itself prone to attack and a
        vulnerability in it could be quite disastrous but the benefits far
        outweigh the risks.
      \end{solution}

    \section*{Introduction to Modern Cryptography Chapter 3}
    \SetQuestionNumber{21}
    \question Let $\prod_1 = ({Gen}_1, {Enc}_1, {Dec}_1)$ and $\prod_2 =
      ({Gen}_2, {Enc}_2, {Dec}_2)$ be two encryption schemes for which it is
      known that at least one is CPA-secure. The problem is that you don't know
      which one is CPA-secure and which one may not be. Show how to construct an
      encryption scheme $\prod$ that is guaranteed to be CPA-secure as long as
      at least one of $\prod_1$ or· $\prod_2$ is CPA-secure. Try to provide a
      full proof of your answer.

      % Hint: Generate two plaintext messages from the original plaintext so
      % that knowledge of either one of the parts reveals nothing about the
      % plain­ text, but knowledge of both does yield the original plaintext.

      \begin{solution}
        Not sure whether this is sound.

        Let $X$ be a random sequence of bits that is the same length as the
        message. Let $Y = m \oplus X$. Neither $X$ or $Y$ reveal anything about
        the plaintext individually. These can then be put through the 2 schemes
        individually. 
      \end{solution}

  \end{questions}
\end{document}
