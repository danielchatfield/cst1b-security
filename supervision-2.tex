\documentclass{supervision}
\usepackage{course}

% Exercises: 10, 12, 13, 15, 18, 19, 24, 25 and 27.

\Supervision{2}
\begin{document}
  \begin{questions}
    \SetQuestionNumber{10}
    \question In the CBC mode of operation, the initial vector (IV) is chosen
      uniformly at random, using a secure source of random bits. Show that CBC
      would not be CPA secure if the initial vector could be anticipated by the
      adversary, for example because it is generated instead using a counter or
      a time-stamp.

    \SetQuestionNumber{12}
    \question A sequence of plaintext blocks $M_1, \ldots , M_8$ is encrypted
      using DES into a sequence of ciphertext blocks. Where an IV is used, it
      is numbered $C_0$. A transmission error occurs and one bit in ciphertext
      block $C_3$ changes its value. As a consequence, the receiver obtains
      after decryption a corrupted plaintext block sequence $M_1', \ldots ,
      M_8'$. For the discussed modes of operation (ECB, CBC, CFB, OFB, CTR),
      how many bits do you expect to be wrong in each block $M_i'$?

      \textit{Hint: You may find it helpful to draw decryption block diagrams}

    \question Your opponent has invented a new stream cipher mode of operation
      for DES. He thinks that OFB could be improved by feeding back into the
      key port rather than the data port of the DES chip. He therefore sets
      $R_0 = K$ and generates the key stream by $R_{i+1} = E_{R_i}(R_0)$. Is
      this better or worse than OFB?

    \SetQuestionNumber{15}
    \question Show that CTR mode is not CCA secure.

    \SetQuestionNumber{18}
    \question The runtime of the usual algorithm for comparing two strings is
      proportional to the length of the identical prefix of the inputs. How and
      under which conditions might this help an attacker to guess a password?

    \question
      \begin{parts}
        \part Describe a cryptographic protocol for a prepaid telephone chip
          card that uses a secure 64-bit MAC function Mac implemented in the
          card. In this scheme, the public telephone needs to verify not only
          that the card is one of the genuine cards issued by the phone
          company, but also that its value counter $V$ has been decremented by
          the cost $C$ of the phone call. Assume both the card and the phone
          know in advance a shared secret $K$. There is no encryption or
          decryption function on the phone or card and the protocol must be
          performed without contacting the phone company.

          \textit{Hint: Protocol equations may make your answer clearer.}

        \part Explain the disadvantage of using the same secret key $K$ in all
          issued phone cards and suggest a way around this.

      \end{parts}

    \SetQuestionNumber{24}
    \question Which of the Unix commands that you know or use are setuid root,
      and why?

    \question What Unix mechanisms could be used to implement capability based
      access control for files? What is still missing?

    \SetQuestionNumber{27}
    \question How can you implement a Clark-Wilson policy under Unix?
  \end{questions}
\end{document}
