\documentclass{supervision}
\usepackage{course}

% Exercises: 10, 12, 13, 15, 18, 19, 24, 25 and 27.

\Supervision{2}
\begin{document}
  \begin{questions}
    \SetQuestionNumber{10}
    \question In the CBC mode of operation, the initial vector (IV) is chosen
      uniformly at random, using a secure source of random bits. Show that CBC
      would not be CPA secure if the initial vector could be anticipated by the
      adversary, for example because it is generated instead using a counter or
      a time-stamp.

      \begin{solution}
        If the IV is predictable then an attacker can confirm whether a guessed
        plaintext ($M_{guess}$) is correct. Suppose they have some ciphertext,
        $C$ that they know was encrypted using an IV $IV_0$, they can check
        their guess by submitting $M_{guess} \oplus IV_1 \oplus IV_0$, when
        this gets XORed with the next IV ($IV_1$) it cancels it out and results
        in the same bits as if it was XORed with just $IV_0$.

        This is used in the SSL BEAST attack, an attacker that could get a
        browser to make requests over https can craft the request such that the
        block is something like ``Cookie: session=X'' where $X$ is a character
        the attacker does not know, with a predictable IV an attacker only has
        to try each of the possibilities and then adapt the request so that the
        block is ``ookie: session=XY'', repeating the process for each
        character.
      \end{solution}

    \SetQuestionNumber{12}
    \question A sequence of plaintext blocks $M_1, \ldots , M_8$ is encrypted
      using DES into a sequence of ciphertext blocks. Where an IV is used, it
      is numbered $C_0$. A transmission error occurs and one bit in ciphertext
      block $C_3$ changes its value. As a consequence, the receiver obtains
      after decryption a corrupted plaintext block sequence $M_1', \ldots ,
      M_8'$. For the discussed modes of operation (ECB, CBC, CFB, OFB, CTR),
      how many bits do you expect to be wrong in each block $M_i'$?

      \textit{Hint: You may find it helpful to draw decryption block diagrams}

      \begin{solution}
        \begin{description}
          \item[ECB] The only block that would contain incorrect bits is $M_3$,
            the number of bits that would be flipped would be random, and thus
            the expectation is that half the bits (32) would be flipped.

          \item[CBC] The decryption of blocks $M_1$ and $M_2$ are independent
            of the value of $C_3$ and thus won't contain any incorrect bits.

            In the decryption of block $M_3$, $C_3$ goes through a pseudo-random
            permutation function and thus the expected number of flipped bits
            is half of them (32 bits).

            For $M_4$, $C_3$ is XORed with the decryption after being put
            through the pseudo-random permutation function, and thus a single
            bit will be flipped.

            For $M_5$ through $M_8$ the decryption is independent again and thus
            no flipped bits will occur.

          \item[CFB] $M_1$, $M_2$, and $M_5$ through $M_8$ will have no flipped
            bits. $M_3$ will have a single flipped bit. $M_4$ will be expected
            to have 32 flipped bits.

          \item[OFB] A single bit will be flipped in $M_3$, all others will have
            no flipped bits.

          \item[CTR] A single bit will be flipped in $M_3$, all others will have
            no flipped bits.
        \end{description}
      \end{solution}

    \question Your opponent has invented a new stream cipher mode of operation
      for DES. He thinks that OFB could be improved by feeding back into the
      key port rather than the data port of the DES chip. He therefore sets
      $R_0 = K$ and generates the key stream by $R_{i+1} = E_{R_i}(R_0)$. Is
      this better or worse than OFB?

      \begin{solution}
        This is worse, for a start it removes the separate random IV as now
        the key is set to $R_0$.

        In addition to this, whilst $E_x$ is a pseudo-random permutation
        function, this scheme limits the possible values it can act on.

        Given a key $K$, $R_1 = E_K(K)$, $R_2 = E_{E_K(K)}(K)$.
      \end{solution}

    \SetQuestionNumber{15}
    \question Show that CTR mode is not CCA secure.

      \begin{solution}
        An adversary $\mathcal{A}$ can choose messages $M_0, M_1 \in
        \{ 0,1 \} ^m $. The challenger then computes $C \leftarrow Enc_K(M_b)$.
        The adversary then XORs $C$ against a random block $X$ and gets the
        challenger to decrypt it. The result, will be one of the original
        messages xored with $X$ and thus the adversary can output $b$.
      \end{solution}

    \SetQuestionNumber{18}
    \question The runtime of the usual algorithm for comparing two strings is
      proportional to the length of the identical prefix of the inputs. How and
      under which conditions might this help an attacker to guess a password?

      \begin{solution}
        If a system uses naïve string comparison to check whether a password is
        correct then an attacker can try each letter in the alphabet over and
        over again until one has a statistically significant lower average
        running time (thus indicating that the comparison moved on to the next
        character).

        This timing attack can be used to recover each character in sequence
        until the entire password has been recovered.
      \end{solution}

    \question
      \begin{parts}
        \part Describe a cryptographic protocol for a prepaid telephone chip
          card that uses a secure 64-bit MAC function Mac implemented in the
          card. In this scheme, the public telephone needs to verify not only
          that the card is one of the genuine cards issued by the phone
          company, but also that its value counter $V$ has been decremented by
          the cost $C$ of the phone call. Assume both the card and the phone
          know in advance a shared secret $K$. There is no encryption or
          decryption function on the phone or card and the protocol must be
          performed without contacting the phone company.

          \textit{Hint: Protocol equations may make your answer clearer.}

          \begin{solution}
            Every card has a unique ID which is not writeable, all
            communication to and from the card except the initial handshake
            includes this ID.

            % TODO

            If the restriction on being able to contact some central system is
            relaxed then the system can be designed in a way that is resilient
            even in the presence of a leaked key. At a fixed time interval
            (could be every minute or once a day) the telephone synchronizes
            with a central system, the central system checks that the balance
            on all cards seen in the time period is consistent with what it has
            seen before. If it is not consistent then it blacklists the card ID.
            This is how Oyster cards work, if you boost your oyster balance on
            the card using a compromised key then the card will be blacklisted
            before you reach your destination.
          \end{solution}

        \part Explain the disadvantage of using the same secret key $K$ in all
          issued phone cards and suggest a way around this.

          \begin{solution}
            The disadvantage is that if someone compromised the key on their
            card, by bruteforce or other means then they have compromised all
            other cards as well. You could get around this by having a
            pseudo-random function that generates a key from an ID and then
            each card will have a unique key.
          \end{solution}

      \end{parts}

    \SetQuestionNumber{24}
    \question Which of the Unix commands that you know or use are setuid root,
      and why?

      \begin{solution}
        % TODO
      \end{solution}

    \question What Unix mechanisms could be used to implement capability based
      access control for files? What is still missing?

      \begin{solution}
        % TODO
      \end{solution}

    \SetQuestionNumber{27}
    \question How can you implement a Clark-Wilson policy under Unix?

      \begin{solution}
        % TODO
      \end{solution}
  \end{questions}
\end{document}
